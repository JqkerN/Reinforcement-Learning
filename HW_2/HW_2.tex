 \documentclass{article}
\usepackage[utf8]{inputenc}
\usepackage{float}
\usepackage{amsmath}
\usepackage[labelsep=period]{caption} % Figur. istället för Figur:
\usepackage{relsize}
\usepackage{multirow}
\usepackage[export]{adjustbox}
\usepackage{subcaption}
\usepackage{wrapfig}
\usepackage{hyperref}
\usepackage{array}
\usepackage{amsmath}
\usepackage{geometry}
\usepackage{parskip}

\title{Reinforcement Learning\\
       Homework 1}

\author{Ilian Corenliussen, 950418-2438, ilianc@kth.se\\ 
        Daniel Hirsch, 960202-5737, dhirsch@kth.se}
\date{November 2020}

\begin{document}

\maketitle

\section{Q-learning and SARSA}
\subsection*{a)}
\textbf{Q-learning Algorithm:} \begin{equation}
 Q^{t+1}(s_t,a) = Q^t(s_t,a) + \alpha \left(R(s_t,a) + \lambda \max_{a \in A}\{Q^t(s_{t+1}, a)\} - Q^t(s_t,a) \right)\end{equation}

We need to calculate the reward for t=2 and the selected action. This can be solved by solving the equation (1). Where $Q^2(B,c)=60$ and by inspecting the matrix we get that the action in $t=2$ is $a$ and for $t=1$ are action $c$. This gives us:
\begin{equation*}
    R_2(A,a) =  \frac{1}{\alpha}(Q^{2}(A,a) - Q^1(A,a) ) - \lambda Q^1(B, a) + Q^1(A,a) = 10( 11 - 0) - 0.5\cdot60 + 0 = 110-30= 80
\end{equation*}
\begin{equation*}
    R_1(B,c) =  \frac{1}{\alpha}(Q^{2}(B,c) - Q^1(B,c) ) - \lambda Q^1(A, a) + Q^1(B, c) = 10( 60 - 0) - 0.5\cdot 0 + 0 = 600
\end{equation*}
\textbf{ANSWER:} (B,c,600);(A,a,80)

\subsection*{b)}
Using equation (1) for the $Q^3-Q^7$ to get the Q-values for the 7th iteration.
\begin{equation*}
     Q^{3}(B,a) = 0 + 0.1(100 + 0.5\cdot11 - 0 ) = 10.6
\end{equation*}
\begin{equation*}
     Q^{4}(A,b) = 0 + 0.1(60 + 0.5\cdot60 - 0 ) = 9
\end{equation*}
\begin{equation*}
     Q^{5}(B,c) = 60 + 0.1(70 + 0.5\cdot0 -60 ) = 61
\end{equation*}
\begin{equation*}
     Q^{6}(C,b) = 0 + 0.1(40 + 0.5\cdot11 - 0 ) = 4.6
\end{equation*}
\begin{equation*}
     Q^{7}(A,a) = 11 + 0.1(20 + 0.5\cdot4.6 -11 ) = 12.3
\end{equation*}
\textbf{ANSWER:} Which gives the Q-values:
\begin{equation*}
    Q^7=
    \left[
    \begin{array}{ccc}
         12.1 & 9 & 0 \\
         10.6 & 0 & 61 \\
         0 & 4.6 & 0 \\
    \end{array}
    \right]
\end{equation*}
\subsection*{c)}
The greedy policy at the 7th iteration will choice the action with the higest value, i.e. $\pi(A)=a$, $\pi(B)=c$, and $\pi(C)=b$.

\textbf{ANSWER:} $\pi(A)=a$, $\pi(B)=c$, and $\pi(C)=b$.

\subsection*{d)}
\textbf{SARSA Algorithm:} \begin{equation}
 Q^{t+1}(s_t,a) = Q^t(s_t,a) + \alpha \left(R(s_t,a) + \lambda Q^t(s_{t+1}, a_{t+1}) - Q^t(s_t,a) \right)\end{equation}
We calculated $Q^1-Q^7$ to get the Q-values at the 7th iteration by SARSA algorithm. 
\begin{equation*}
     Q^{1}(B,c) = 0 + 0.1(600 + 0.5\cdot0 - 0 ) = 60
\end{equation*}
\begin{equation*}
     Q^{2}(A,a) = 0 + 0.1(80 + 0.5\cdot0 - 0 ) = 8
\end{equation*}
\begin{equation*}
     Q^{3}(B,a) = 0 + 0.1(100 + 0.5\cdot0 - 0 ) = 10
\end{equation*}
\begin{equation*}
     Q^{4}(A,b) = 0 + 0.1(60 + 0.5\cdot60 - 0 ) = 9
\end{equation*}
\begin{equation*}
     Q^{5}(B,c) = 60 + 0.1(70 + 0.5\cdot0 - 60 ) = 61
\end{equation*}
\begin{equation*}
     Q^{6}(C,b) = 0 + 0.1(40 + 0.5\cdot8 - 0 ) = 4.4
\end{equation*}
\begin{equation*}
     Q^{7}(A,a) = 8 + 0.1(20 + 0.5\cdot0 -8 ) = 9.2
\end{equation*}

\textbf{ANSWER:}
\begin{equation*}
    Q^7=
    \left[
    \begin{array}{ccc}
        9,2 & 9 & 0 \\
         10 & 0 & 61 \\
         0 & 4.4 & 0 \\
    \end{array}
    \right]
\end{equation*}

\subsection*{e)}
The greedy policy at the 7th iteration will choice the action with the higest value, i.e. $\pi(A)=a$, $\pi(B)=c$, and $\pi(C)=b$.

\textbf{ANSWER:} $\pi(A)=a$, $\pi(B)=c$, and $\pi(C)=b$.


\end{document}
